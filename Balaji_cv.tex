\documentclass[11pt,a4paper,sans]{moderncv}

% moderncv themes
\moderncvstyle{banking} % casual, classic, banking, oldstyle and fancy
\moderncvcolor{blue}

\usepackage[utf8]{inputenc}
\usepackage[scale=0.75]{geometry}
%\usepackage{hyperref}

\patchcmd{\makehead}
  {\hfil}
  {\hspace*{0.15\textwidth}}
  {}
  {}
\patchcmd{\makehead}
  {\setlength{\makeheaddetailswidth}{0.8\textwidth}}
  {\setlength{\makeheaddetailswidth}{0.67\textwidth}}
  {}
  {}
\patchcmd{\makehead}
  {\\[2.5em]}
  {\hfil\raisebox{-.7cm}{\framebox{\includegraphics[width=\@photowidth]{\@photo}}}\\[2.5em]}
  {}
  {}

% personal data
\name{Balaji}{R}
\title{Curriculum Vitae}
\address{Indian Institute Of Science}{Bangalore - 560012}{India}
\phone[fixed]{+91~93848~08412}
\email{balajiwork01@gmail.com}
\photo[64pt][100pt]{cv_photo}
%\cvitem{Linkedin}{\href{http://www.linkedin.com/in/balaji-ram-b45061257}{\textcolor{blue}{Balaji Ram}}}


\extrainfo{\faGithub\href{http://github.com/blackscreen-whitetext}{\textcolor{blue}{blackscreen-whitetext}}\ \faLinkedin{\href{http://www.linkedin.com/in/balaji-ram-b45061257}{\textcolor{blue}{Balaji Ram}}}}
\extrainfo{Github Pages: \href{https://blackscreen-whitetext.github.io/}{\textcolor{blue}{blackscreen-whitetext.github.io}}}
\begin{document}
\makecvtitle

\section{Projects:}
\cvitem{(Ongoing) Portflio Hedging Using Options}{Using Linear Programs to minimize CVaR with Prof. Shashi Jain}
\cvitem{(Ongoing) IGEM, Synthetic Biology Competition}{Leveraging Generative AI to synthesise protein sequences(MiRNAs) for designing therapeutics}
\section{Online Certifications:}
\cvitem{Generative AI with LLMs}{\href{https://github.com/blackscreen-whitetext/Course-certs/blob/main/Generative_AI_With_LLMs_Andrew_NG/Balaji_LLMs.pdf}{\textcolor{blue}{Course Certificate}}} \vspace{5pt}
\begin{itemize}
  \item Learnt About the various classes of transformer architectures and the usecases of various LLMs(GPT, BERT, ELMo, BART, T5, Llama)
\item Used Amazon Sagemaker and experimented these concepts with FLAN-T5: Prompt Engineering, Prompt Tuning, Fine-Tuning LLMs(Parameter Efficient Methods(PEFT)), Reinforcement Learning From Human Feedback, Retrieval Augmented Generation
\item Learnt about program aided models like copilot and the ReAct paper that uses chain of thought reasoning and Action Words to make the LLM generate better outputs.
\end{itemize}\vspace{5pt}
\cvitem{Short course On LangChain}{\href{https://learn.deeplearning.ai/courses/langchain}{\textcolor{blue}{Course Page}}} \vspace{5pt}
\cvitem{Natural Language Processing Specialization}{\href{https://github.com/blackscreen-whitetext/Course-certs/blob/main/Natural_Language_Processing_DeepLearning/NLP_Specialization_Balaji.pdf}{\textcolor{blue}{Course Certificate}}}
\begin{itemize}
  \item Implemented Naive Bayes Classifier and Hidden Markov Models for Parts Of Speech Tagging.
  \item Implemented An Autocomplete System using N-grams
  \item Implemented a Continuous Bag Of Words Model for Word Embeddings
  \item Used Deep Neural Networks For Sentiment Analysis
  \item Used RNNs, GRUs, LSTMs in Named Entity Recognition
  \item Used attention with LSTM for Neural Machine Translation
  \item Learnt about metrics to evaluate language models like ROUGE, BLEU, perplexity and benchmarks like GLUE, SuperGLUE to compare language models.
  \item Implemented a transformer for text summarization.
  \item Used the huggingface transformers library for question answering
\end{itemize}\vspace{5pt}
\cvitem{Generative Adversarial Networks Specialization}{\href{https://github.com/blackscreen-whitetext/Course-certs/blob/main/Generative_Adversarial_Networks_DeepLearning_AI/Generative_Adversarial_Networks_Specialisation_Balaji.pdf}{\textcolor{blue}{Course Certificate}}}
\textbf{GAN Implementations:} DCGAN, CycleGAN,W-GAN, Pix2Pix, StyleGAN, Data Augmentation using GANs, Conditional And Controllable Generation\\ \\
\vspace{5pt}
\cvitem{Deep Learning Specialization By Andrew NG}{\href{https://github.com/blackscreen-whitetext/Course-certs/blob/main/Deep_Learning_AndrewNG/Deep_Learning_Specialisation_Balaji.pdf}{\textcolor{blue}{Course Certificate}}}\vspace{2pt}
\textbf{Selected Concepts:} CNNs, transfer learning, RNNs, GRUs, LSTMs, Attention Models, Word Embeddings(word2vec,GLoVE), Transformers\\
\textbf{Selected Applications:} Object detection(YOLO), Image Recognition, Image Segmentation(UNet) Speech Translation\\ \\
\cvitem{Short Course On Diffusion Models}{\href{https://learn.deeplearning.ai/courses/diffusion-models/}{\textcolor{blue}{Course Page}}}\vspace{5pt}
\cvitem{Machine Learning Specialization By Andrew NG}{\href{https://github.com/blackscreen-whitetext/Course-certs/blob/main/Machine_Learning_AndrewNG/Machine_Learnning_Specialization_Balaji.pdf}{\textcolor{blue}{Certificates}}}\vspace{2pt}
\textbf{Selected Concepts:} Anomaly Detection, Recommender Systems, Deep-Q Reinforcement Learning, Support Vector Machines, K-means Clustering, PCA.\\ \\
%I learnt how to implement forward and backward propagation from scratch, deep learning frameworks such as tensorflow. I implemented CNNs and used them for object detection, and in an image recognition system, used pretrained models with a common practical technique called transfer learning. Also i learnt how to handle sequence data with RNNs, GRU, LSTM with the application in sentiment analysis, speech translation etc. I came across vast field of natural language processing with word embeddings using word2vec, GLoVe. Finally attention models including an in depth understanding of the transformer architecture, and have read about the BERT model.\\ \\
\cvitem{Kaggle certifications:}{\href{https://github.com/blackscreen-whitetext/Course-certs/tree/main/kaggle}{\textcolor{blue}{Certificates}}}%\vspace{2pt}
%\cvitem{Financial Markets By Robert Shiller}{\href{https://www.coursera.org/learn/financial-markets-global/}{\textcolor{blue}{Coursera Page}}}
\section{Key UG Courses:}
\cvitem{Artificial Intelligence And Machine Learning:}{} %\vspace{2pt}
\begin{itemize}
   \item Wrote a term paper and presentation to explore diffusion models' capabilities to generate images. 
  \item Learnt to solve convex optimization problems using cvxopt in python.
\end{itemize}
\cvitem{Data Structures And Algorithms:}{Implemented algorithms for various problems in C++ and python.} 
\cvitem{Probability And Statistics:}{Learnt statistical inference in MATLAB.}
\cvitem{Numerical Analysis:}{Implemented methods to numerically solve ODEs,PDEs in python.}
\cvitem{Algorithms And Programming:}{Learnt problem solving in C}
\cvitem{Computer Systems:}{Learnt about operating systems, hardware, memory}
\section{Skills}
\cvitem{Languages}{Python, C, C++ MATLAB, \LaTeX, R, SQL, HTML, CSS, JavaScript(React)}
\cvitem{ML Frameworks:}{Tensorflow, Keras, Pytorch}
\cvitem{LLM:}{Huggingface Transformers, (Models:T5,BERT,Llama) Coding Platforms: AWS Sagemaker Jumpstart, Google Colab, Jupyter Notebook}
\cvitem{Libraries}{Numpy, scipy, Pandas, Matplotlib, Seaborn, Scikit-learn,cvxopt}
\cvitem{Tools}{Git, Linux, VS Code}
\section{Education}
\cventry{2022--current}{Bachelors Of Technology In Mathematics And Computing}{Indian Institute Of Science}{Bangalore}{\textit{CGPA}--9.0/10.0}{Secured admission through JEE Advanced\\ Expected to graduate in 2026}\vspace{10pt}
\cventry{2007--2022}{School}{PSBB KK Nagar}{Chennai}{}{\textit{12th Percentage}--96.8\% \\ \textit{10th Percentage}--95\%}
\section{Volunteering:}
\begin{itemize}
  \item \textbf{Team Vicharaka:} Currently on the team of students building a mars rover for the university rover challenge.
  \item \textbf{Databased(The undergraduate Computer Science club):} Explained prompt engineering to students as part of our club on open day. 
  \item \textbf{Counselling:} Volunteered to be a part of the Q\&A session of the counselling process for the incoming batch of students.
\end{itemize}
\section{Achievements}
\cvitem{JEE Advanced}{AIR 225}
\cvitem{KVPY SA}{AIR 175}
\cvitem{JEE Mains}{AIR 1005}




\end{document}
